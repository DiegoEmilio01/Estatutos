\documentclass[letterpaper,11pt]{article}
\usepackage[spanish]{babel}
\usepackage[utf8]{inputenc}
\usepackage[usenames,dvipsnames]{color}
\usepackage[margin=1in]{geometry}
\usepackage[usenames,dvipsnames]{color}
\usepackage[T1]{fontenc}
\usepackage{lmodern}
\usepackage{textcomp}
\usepackage[exercise]{amsthm}
\usepackage{hyperref}
\usepackage{amsmath}
\usepackage{amsthm}
\usepackage{fancyhdr}
\usepackage{titlesec}
\hypersetup{colorlinks= true, linkcolor=blue}
\spanishdecimal{.}
\tolerance=1000
\hyphenpenalty=1000
\setlength{\parindent}{0in}
\setlength{\parskip}{0.1in}
\setlength\textheight{8.2in}
\setlength\topmargin{-1in}
\lhead{\sc Estatutos Centro de Alumnos de Ingeniería\\Pontificia Universidad Católica de Chile}
\chead{}
\rhead{}
\lfoot{}
\cfoot{\thepage}
\rfoot{}
\fancypagestyle{plain}{%
\fancyhf{}
\cfoot{\thepage}
\lfoot{}
\renewcommand{\headrulewidth}{0pt}}
\renewcommand{\headrulewidth}{1 pt}
\renewcommand{\footrulewidth}{0 pt}
\headheight=75pt

\theoremstyle{plain}
\newtheorem{art}{Art.} %Para que salga Art.+Nro
\newtheorem{art_trans}{Art.}

\renewcommand \thesection{\Roman{section}} %Enumerar secciones con números romanos
%\titleformat{\section}[display]{\scshape \Large}{Título \thesection:}{}{}[]
%Enumeraciones
\renewcommand{\theenumi}{\Alph{enumi}}
\renewcommand{\labelenumi}{\theenumi.}
\renewcommand{\theenumii}{\arabic{enumii}}
\renewcommand{\labelenumii}{(\theenumii)}
\renewcommand{\theenumiii}{\roman{enumiii}}
\renewcommand{\labelenumiii}{\theenumiii.}
\renewcommand{\theenumiv}{(\alph{enumiv})}
\newcommand{\HRule}{\rule{\linewidth}{0.5mm}}
\newcommand{\aref}[1]{\hyperref[#1]{\ref*{#1}}}
\newcommand{\aaref}[2]{\hyperref[#2]{\ref*{#1}, letra \ref*{#2}}}
\makeatletter \renewcommand\p@enumii{\theenumi, } \makeatother
\makeatletter \renewcommand\p@enumiii{\theenumii, } \makeatother
\makeatletter \renewcommand\p@enumiv{\theenumiii, } \makeatother

%Formalidades respecto al título, autor y fecha de modificación
\title{Estatutos}
\author{Centro de Alumnos de Ingeniería}
\date{Noviembre 2015}
%Inicio del documento
\begin{document}

	\thispagestyle{plain}
	\vspace*{-75pt}

	\begin{center}
		\begin{Large}
			{\bf
			ESTATUTOS DEL CENTRO DE ALUMNOS DE INGENIERÍA

			PONTIFICIA UNIVERSIDAD CATÓLICA DE CHILE
			}
		\end{Large}

		\vspace*{30pt}

	\end{center}

	\section{Finalidades y miembros}\label{finalidades}

		\begin{art}\label{interpretacionEstatuto}
			De la interpretación del Estatuto.

			El presente estatuto se interpretará en virtud de las normas de interpretación del Código Civil\footnote{Código Civil: \url{http://www.leychile.cl/Navegar?idNorma=172986}}.

			En caso de vacío, el Comité Ejecutivo dictará un Oficio Interpretativo, que deberá ser aprobado en sesión Ordinaria o Extraordinaria del Consejo Generacional por mayoría absoluta. Su resolución será obligatoria desde este caso en particular en adelante.

			Este Oficio Interpretativo deberá, asimismo, anexarse al Estatuto con todas las medidas de publicidad requeridas por éste.
		\end{art}

		\begin{art}\label{representacionCAi}
			El Centro de Alumnos de Ingeniería, CAi, es un organismo que representa a todos los alumnos de Pregrado de la Escuela de Ingeniería de la Pontificia Universidad Católica de Chile, así como a los alumnos de Magíster en Ingeniería, Magíster en Ciencias de la Ingeniería o Doctorado de esta misma Unidad Académica.
		\end{art}

		\begin{art}\label{finalidadesCAi}
			El CAi es un organismo autónomo respecto a cualquier otra organización, tanto estudiantil como universitaria. En esencia, el CAi debe velar por los intereses del grupo estudiantil que representa. Para ello estructurará y conducirá el aporte de los estudiantes de Ingeniería a la Universidad y a la sociedad. Para cumplir este fin deberá abocarse, a través de sus organismos directivos, a las siguientes funciones:

			\begin{enumerate}

				\item \label{gremial}\textsc{\sc En el aspecto gremial:}
					\begin{enumerate}
						\item Organizar, coordinar y supervisar la participación estudiantil en los organismos colegiados de la Universidad. Buscar medios para lograr una verdadera comunicación estudiantil y profesional de la Escuela.
					\end{enumerate}

				\item \label{social} \textsc{En el aspecto social:}
					\begin{enumerate}
						\item Lograr que el conocimiento de la realidad social del país sea parte fundamental en la formación de los estudiantes de Ingeniería.

						\item \label{bienestar_social} Estimular y apoyar actividades destinadas a lograr el bienestar social de los estudiantes y, en lo posible, de la comunidad en que está inserta.
					\end{enumerate}

				\item \label{cultural}\textsc{En el aspecto cultural y de iniciativas estudiantiles:}
					\begin{enumerate}
						\item Promover una sólida formación cultural en el alumno y realizar actividades conducentes a ella.

						\item Estimular la creación y el desarrollo de iniciativas estudiantiles de diversa índole, por medio de Fondos Concursables y apoyo directo.
					\end{enumerate}

			\end{enumerate}
		\end{art}

		\newpage

		\begin{art}\label{organismosCAi}
			Los organismos directivos del CAi son los siguientes:
			\begin{enumerate}
				\item \label{jerarquia} \textsc{Comité Ejecutivo} Encabeza al CAi. Está formado por ocho miembros. En orden jerárquico, este está formado por:
					\begin{enumerate}
						\item Presidente.
						\item Primer Vicepresidente.
						\item Segundo Vicepresidente.
						\item Secretario General.
						\item Jefe de Finanzas.
						\item Coordinador de Áreas.
						\item Jefe de Docencia
						\item Jefe de Investigación
					\end{enumerate}
				El Presidente podrá ampliar el número y las funciones y atribuciones de los miembros del Comité Ejecutivo dentro de sus propias facultades.

				\item \textsc{Consejería Académica de Pregrado, conformada por}:
					\begin{enumerate}
						\item Consejero Académico de Pregrado, en adelante Consejero Académico
						\item Subconsejero Académico de Pregrado, en adelante Subconsejero Académico
					\end{enumerate}

				\item \textsc{Consejería Académica de Postgrado, conformada por}:
					\begin{enumerate}
						\item Consejero Académico de Postgrado, en adelante Consejero de Postgrado
						\item Subconsejero Académico de Postgrado, en adelante Subconsejero de Postgrado
					\end{enumerate}

				\item \textsc{Comité Generacional} Está formado por tres representantes de cada generación, cuyo año de ingreso sea menor al de la quinta generación y tres representantes de la sexta o superior. Los anteriores serán nombrados en adelante como Delegados Generacionales.

				\item \textsc{Comité Académico} Está formado por un representante de cada major, que debe estar inscrito en el major respectivo y que debe estar cursando al menos la tercera generación; por un representante de cada departamento, centro o especialidad, que debe estar cursando una mención en ese departamento o realizando dicha especialidad, con un máximo de un representante por departamento o centro; por 7 representantes de postgrado (3 votos), de cada una de las áreas del programa; y un Representante de College. Las áreas son las siguientes:
					\begin{enumerate}
						\item Ingeniería Civil (Gestión de la Construcción, Estructural y Geotecnia, Hidráulica y Ambiental
						\item Industrial y de Transporte e Instituto de Matemática
						\item Mecánica y Metalurgia 
						\item Eléctrica
						\item Ciencias de la Computación
						\item Química y Bioprocesos e Instituto de Ingeniería Biológica y Médica)
					\end{enumerate}

				\item \textsc{Consejo Generacional} Es la máxima instancia de representación. Formado por el Comité Generacional, el Comité Ejecutivo, el Consejero Académico y el Consejero de Postgrado. Es presidido por el Presidente del CAi.

				\item \textsc{Consejo Académico} Formado por el Comité Académico, el jefe de Docencia y el Jefe de Investigación del Comité ejecutivo, y los miembros de las Consejerías Académicas de Postgrado y Pregrado, quien lo preside.

				\item \textsc{Comisiones} Las comisiones dependen del Comité Ejecutivo y lo ayudan en su gestión. El Comité Ejecutivo puede formar las comisiones que estime conveniente durante el período de su gestión, informando de ello a los Consejos. A modo de referencia, una lista sugerida de las posibles comisiones es la siguiente:
					\begin{enumerate}
						\item Auspicios
						\item Comunicaciones
						\item Comunidad
						\item Cultura
						\item Deportes
						\item Docencia
						\item Emprendimiento
						\item Novatos
						\item Política y Actualidad
						\item Proyectos
						\item Responsabilidad Social
						\item Sustentabilidad
						\item Vida Universitaria
					\end{enumerate}

				\item \textsc{Coordinadores Generales} Alumno regular de la Escuela de Ingeniería que es designado por el Comité Ejecutivo para apoyar o liderar algún área de trabajo del CAi. La designación de uno o más Coordinadores Generales debe ser informado al Consejo Generacional, al igual que eventuales reemplazos o renuncias. No es miembro del Comité Ejecutivo.
			\end{enumerate}
		\end{art}

		\begin{art}\label{funcionesEjecutivo1}
			Las funciones del Comité Ejecutivo son:
			\begin{enumerate}
				\item Desarrollar una gestión de excelencia en favor del Alumnado y la Comunidad de la Escuela de Ingeniería, de acuerdo a lo presentado en su programa de postulación, de manera no vinculante. Para ello, podrá crear comisiones y áreas de trabajo en las que podrá nombrar y remover a sus respectivos Jefes de Comisión; como también podrá designar Coordinadores Generales.

				\item Entregar a todos los miembros del Consejo Generacional y del Consejo Académico una copia de los estatutos vigentes al inicio del período.
				
				\item Asumir la última responsabilidad de las actuaciones en las cuales los organismos dependientes del Comité Ejecutivo se vean involucrados, liberándose de esta responsabilidad al desautorizar públicamente y antes de que ocurra el acto que no cuenta con apoyo oficial.
				
				\item Mantener archivo oficial, redactar y despachar la correspondencia.
				
				\item Velar por el cumplimiento del presente cuerpo de estatutos, además de desarrollar sus actividades de acuerdo a los reglamentos y protocolos que de él emanan.
			\end{enumerate}
		\end{art}

		\begin{art}%\label{funcionesEjecutivo2}
			Las funciones y atribuciones mínimas de los miembros del Comité Ejecutivo son las que se numeran a continuación. Esta lista no es exhaustiva, y sin perjuicio de ella los miembros del Comité Ejecutivo colaborarán cumpliendo con los encargos que el Presidente les encomiende.
			\begin{enumerate}
				\item \label{funciones_presidente}\textsc{Presidente}
					\begin{enumerate}
						\item Presidir y convocar las reuniones del Comité Ejecutivo y del Consejo Generacional. Sin perjuicio de la posibilidad de auto convocatoria de éste último, dispuesto en el artículo \aref{consejoExtraordinario}.
						
						\item Representar al CAi y sus posturas ante las autoridades de la Escuela, la Universidad, la Federación y otros organismos pertinentes. Se considerará como postura del CAi la que adopte el Consejo Generacional.
						
						\item Dar cuenta pública y detallada de la gestión realizada al término de cada semestre en ejercicio a nombre del Comité Ejecutivo.
						
						\item Definir nuevas funciones y atribuciones a los miembros del Comité Ejecutivo. No podrá atribuir derechos o deberes que se opongan abiertamente a lo señalado en estos estatutos, o que superen sus propias funciones o atribuciones. Cualquier definición de este tipo deberá ser informada a los Consejos.
						
						\item Definir la división de labores que corresponde a cada vicepresidente.
						
						\item Dar cuenta pública, en las sesiones ordinarias del Consejo Generacional y del Consejo Académico, de su participación en el Consejo de Federación y en las instancias en las que participa regularmente como representante de los alumnos ante las autoridades de la Escuela.
					\end{enumerate}

				\item \textsc{Primer Vicepresidente}
					\begin{enumerate}
						\item Representará al CAi en los organismos respectivos cuando el Presidente se lo solicitara, o cuando este faltare por motivos de fuerza mayor, con todas las atribuciones del cargo.
						
						\item Coordinará y supervisará la labor de las comisiones y equipos de trabajo dependientes del Comité Ejecutivo. La división de las labores de los dos vicepresidentes quedará a criterio del Presidente en ejercicio.
					\end{enumerate}

				\item \textsc{Segundo Vicepresidente}
					\begin{enumerate}
						\item Coordinará y supervisará la labor de las comisiones y equipos de trabajo dependientes del Comité Ejecutivo. La división de las labores de los dos vicepresidentes quedará a criterio del Presidente en ejercicio.
					\end{enumerate}

				\item \textsc{Secretario General}
					\begin{enumerate}
						\item Presidir y convocar las reuniones del Comité Ejecutivo y del Consejo Generacional. Sin
						perjuicio de la posibilidad de auto convocatoria de éste último, dispuesto en el artículo 45.
						
						\item Llevar control de asistencia y someter a aprobación el acta de cada Consejo, que se dará por aprobada mediante correo electrónico, a menos que un miembro del consejo se oponga. El plazo para oponerse será de 2 días. En caso de rechazo de la primera publicación del acta,
						existirán 3 instancias mediante correo electrónico para llegar a consenso sobre el acta escrita (entiéndase como instancia cada vez que se le envíe a los delegados), en caso de superar esta última, se deberá aprobar en el Consejo siguiente.
						
						\item Publicar actas del Consejo Académico en la página web del Centro de Alumnos, con un plazo máximo de cinco días hábiles desde su aprobación por el Consejo.
						
						\item Coordinar y supervisar el correcto funcionamiento de las comisiones, mediante la coordinación con un encargado de cada comisión.
						
						\item Publicar las cuentas públicas de las comisiones y los delegados en la página del Centro de Alumnos y por otro medio que se considere pertinente para que llegue a los alumnos de la Escuela, en un plazo máximo de cinco días hábiles posterior al consejo en que fueron presentadas.
					\end{enumerate}

				\item \textsc{Jefe de Finanzas}
					\begin{enumerate}
						\item Administrar y llevar una cuenta de todos los ingresos y gastos del CAi, en la cual estos aparezcan convenientemente justificados y dar cuenta de ello una vez al mes en la página web del Centro de Alumnos, y una vez al semestre ante los Consejos Generacional y Académico. Con la publicación mensual de ingresos y gastos, se debe adjuntar la cartola mensual de la cuenta correspondiente a la caja chica del CAi y la glosa explicativa.
						
						\item Dar la información que le fuese solicitada por algún miembro del CAi y que se refiera a la administración y finanzas.
						
						\item Presentar el presupuesto anual del CAi en la sesión ordinaria de los  Consejo Generacional y Académico correspondiente al mes de abril.
						
						\item Presentar el estado del presupuesto anual del CAi en las sesiones ordinarias de los Consejo Generacional y Académico correspondientes al mes de julio y noviembre. En la sesión de noviembre, se invitará al jefe de finanzas electo para el año siguiente.
					\end{enumerate}

				\item \textsc{Coordinador de Áreas}
					\begin{enumerate}
						\item Coordinar y supervisar la labor de las comisiones, velando por el correcto funcionamiento de estas.
					\end{enumerate}

				\item \textsc{Jefe de Docencia}
					\begin{enumerate}
						\item Colaborar en las políticas académicas de la Escuela, trabajando en conjunto con el Consejero Académico.
						
						\item Asistir al Comité de Pregrado.
						
						\item Asistir al Consejo Académico.
					\end{enumerate}

				\item \textsc{Jefe de Investigación}
					\begin{enumerate}
						\item Integrar a las políticas académicas de la Escuela la visión de los alumnos de Postgrado, trabajando en conjunto con el Consejero Académico de Postgrado.
						\item Asistir al Comité de Postgrado.
						\item Asistir al Consejo Académico.
					\end{enumerate}
					
				\item \textsc{Cargos opcionales}
					Las funciones de los cargos opcionales son definidas por el Presidente, tal como se señala en la parte A del presente artículo. 
			\end{enumerate}
		\end{art}

		\begin{art}\label{funcionesCAPregrado}
			Las funciones y atribuciones del Consejero Académico son:
			\begin{enumerate}
				\item Integrar a las políticas académicas de la Escuela la visión de los alumnos, liberando de estas responsabilidades al Presidente del CAi.
				\item Presidir el Consejo Académico.
				\item Representar a los alumnos ante el \emph{CIDEI}, el Consejo de la Escuela de Ingeniería y el Comité de Pregrado, en conjunto con el Presidente de CAi.
				\item Estudiar y desarrollar proyectos y propuestas académicas con la Dirección de la Escuela y, si existiese, la Comisión de Docencia del CAi, preocupándose también de su difusión.
				\item Preparar y ejecutar las defensas de los alumnos respecto de las causales de eliminación, frente a la Comisión de la Escuela.
				\item Asistir a las instancias formales para las que sea llamado junto a sus pares de las demás facultades por el Consejero Superior FEUC, tales como el Consejo Académico FEUC.
				\item Dar cuenta pública, en las sesiones ordinarias del Consejo Generacional y Académico, de su participación en el Consejo Académico FEUC y en las instancias en las que participa regularmente como representante de los alumnos ante las autoridades de la Escuela.
			\end{enumerate}
		\end{art}

		\begin{art}
			Las funciones y atribuciones del Subconsejero Académico son:
			\begin{enumerate}
				\item Colaborar en las políticas académicas de la Escuela, trabajando en conjunto con el Consejero Académico.
				\item Reemplazar al Consejero Académico en el Consejo Académico FEUC, cuando éste no puede asistir.
				\item Reemplazar al Consejero Académico en caso de renuncia, cese anticipado o destitución del mismo.
			\end{enumerate}
		\end{art}

		\begin{art}\label{funcionesCAPostgrado}
			Las funciones y atribuciones del Consejero de Postgrado son:
			\begin{enumerate}
				\item Integrar a las políticas académicas de la Escuela la visión de los alumnos de Postgrado, liberando de estas responsabilidades al Presidente del CAi.
				\item Representar a los alumnos ante el Comité de Postgrado.
				\item Asistir a las instancias formales de representación para las que sea llamado por organismos atingentes a su cargo y rol.
				\item Coordinar con el Comité Ejecutivo la creación y ejecución de una agenda de trabajo orientada a los alumnos de Postgrado de la Escuela de Ingeniería.
				\item Asistir al Consejo Académico
				\item Participar en el Comité de Magíster/Doctorado según el programa al que pertenece (Ya sea al Comité de Magíster, si es del Magíster en Ciencias de la Ingeniería o al Comité de Doctorado si es del Doctorado en Ciencias de la Ingeniería).
				\item Sesionar a los Delegados de Postgrado una vez al mes.
			\end{enumerate}
		\end{art}

		\begin{art}
			Las funciones y atribuciones del Subconsejero de Postgrado son:
			\begin{enumerate}
				\item Integrar a las políticas académicas de la Escuela la visión de los alumnos de Postgrado, trabajando en conjunto con el Consejero Académico de Postgrado.
				\item Reemplazar al Consejero de Postgrado en el Consejo Académico de Postgrado UC, cuando éste no puede asistir.
				\item Reemplazar al Consejero Académico en caso de renuncia, cese anticipado o destitución del mismo.
				\item Participar en el Comité de Magíster/Doctorado según el programa al que pertenece (Ya sea al Comité de Magíster, si es del Magíster en Ciencias de la Ingeniería o al Comité de Doctorado si es del Doctorado en Ciencias de la Ingeniería).
			\end{enumerate}
		\end{art}

		\begin{art}\label{representanteCollege}
			Las funciones y atribuciones del Representante de College son:
			\begin{enumerate}
				\item Representar a los alumnos de College que se encuentran relacionados a la Escuela de Ingeniería principalmente en temas académicos.
				\item Dar cuenta pública de su gestión al menos dos veces durante su período a sus representados. Esta deberá ser presentada en sesión ordinaria del Consejo Académico y difundida por el CAi.
				\item Entregar un documento resumen al momento del cambio de mando, en donde se expliciten los principales proyectos llevados a cabo durante su periodo en ejercicio. 
				\item Crear un programa de trabajo que deberá ser público al inicio de su gestión, y disponible en la página web del CAi a más tardar dos semanas después del haber asumido.
			\end{enumerate}
		\end{art}

		\begin{art}\label{funcionesDelegadosGeneracionales}
			Las funciones y atribuciones de los Delegados Generacionales son:
			\begin{enumerate}
				\item Representar políticamente a sus compañeros.
				\item Velar por la excelencia académica de los cursos que competen a sus representados, principalmente aquellos de ciencias básicas y comunes dictados en esta u otra facultad.
				\item Para los Delegados de la primera y segunda generación, al menos uno de cada generación debe asistir a las sesiones del Consejo Académico, pudiendo alternar esta función entre más de un Delegado a lo largo del año académico.
				\item Crear un programa de trabajo que deberá ser público al inicio de su gestión, y disponible en la página web del CAi a más tardar 2 semanas después de haber asumido.
				\item \label{cuentaPublica} Al menos dos veces durante su período deberán rendir cuenta de las actividades realizadas a sus representados. Estas deberán ser presentadas en sesión ordinaria del Consejo Generacional y también difundida por el CAi  y los delegados por el medio que estimen conveniente.
				\item Entregar un documento resumen al momento del cambio de mando, en donde se expliciten los principales proyectos llevados a cabo durante su período en ejercicio.
			\end{enumerate}
		\end{art}

		\begin{art}\label{funcionesComiteGeneracional}
			Las funciones y atribuciones del Comité Generacional son:
			\begin{enumerate}
				\item Servir de puente de información e inquietudes entre el Consejo Generacional, la Escuela y el estudiantado.
				\item Colaborar en actividades de integración a la Escuela y generar instancias de participación para sus respectivas generaciones.
				\item Decidir los votos ponderados de Ingeniería para el Consejo de Federación los cuales son llevados al mismo por los Consejeros Territoriales.
				\item Colaborar en la comunicación de actividades y procesos del Comité Ejecutivo, además de fiscalizar su gestión.
				\item Desautorizar públicamente una actuación del Comité Ejecutivo, o alguno de sus miembros; del Consejero Académico y del Consejero de Postgrado.
				\item Formular la solicitud de destitución de su cargo a cualquier miembro de los organismos del CAi, por grave infracción a este estatuto o notable abandono de deberes.
			\end{enumerate}
		\end{art}

		\begin{art}\label{funcionesConsejoGeneracional}
			Las funciones y atribuciones del Consejo Generacional son:
			\begin{enumerate}
				\item Pronunciarse sobre los temas políticos que cualquiera de sus miembros someta a su consideración.
				\item Pronunciarse sobre los temas planteados por la Federación de Estudiantes de la Universidad Católica (FEUC) a través de su Consejo de Presidentes o cualquiera de sus instancias.
				\item Someter a plebiscito cualquier decisión que considere pertinente, en caso de presentarse mayoría simple del total de los votos posibles.
				\item Someter a aprobación toda consulta, elección o plebiscito que no contare con el quórum mínimo para su validación.
				\item Velar por el cumplimiento del presente cuerpo de Estatutos.
				\item Destituir de su cargo a todo miembro del Comité Ejecutivo que cumpliere con algunas de las causas establecidas como motivo de cesación de cargo en el presente estatuto, artículo.
				\item  Crear comisiones de trabajo en función de la contingencia o interés de los miembros del Consejo Generacional. La creación de una comisión deberá contar con la aprobación de al menos un quinto de los votos totales del consejo. El rol de estas comisiones será recopilar información y/o elaborar propuestas para luego presentarlas al Consejo Generacional. Las comisiones deben estar formadas por al menos 4 delegados y liderada por uno de estos. \ref{requisitosEjecutivo}.
			\end{enumerate}
		\end{art}

		\begin{art}\label{funcionesDelegadosAcademicos}
			Las funciones y atribuciones de los Delegados Académicos son:
			\begin{enumerate}
				\item En el caso de los Delegados de Major, canalizar los problemas académicos de los alumnos de su Major y de Minors asociados y representarlos ante las instancias académicas correspondientes. En el caso de los Delegados de Especialidad, canalizar los problemas académicos de los alumnos del segundo ciclo de formación de su Departamento y representarlos ante las instancias académicas correspondientes.
				\item Velar por la excelencia académica de los cursos que competen a sus representados, principalmente aquellos que son dictados en sus respectivas disciplinas.
				\item Representar ante el Consejo Académico a los Capítulos Estudiantiles y otros proyectos estudiantiles de índole académica de sus respectivas disciplinas.
				\item Para el caso de los delegados de especialidad, participar de las reuniones del departamento del cual está adscrito como representante.
				\item Para el caso de los delegados de major, participar del  Comité de Programa del major respectivo, y si corresponde, participar de las reuniones de departamento al cual el major está adscrito.
				\item Crear un programa de trabajo que deberá ser público al inicio de su gestión, y disponible en la página web del CAi a más tardar dos semanas después del haber asumido.
				\item Al menos dos veces durante su período deberán rendir cuenta de las actividades realizadas a sus representados. Estas deberán ser presentadas en sesión ordinaria del Consejo Académico y también difundida por el CAi  y los delegados por el medio que estimen conveniente.
				\item Entregar un documento resumen al momento del cambio de mando, en donde se expliciten los principales proyectos llevados a cabo durante su periodo en ejercicio.
				\item Para los Delegados de postgrado, al menos uno de ellos debe asistir a las sesiones del Consejo Generacional, pudiendo alternar esta función entre más de un Delegado a lo largo del año académico. Además, deben sesionar por lo menos una vez al mes para conversar de los diferentes problemas que estén surgiendo en el Programa.
			\end{enumerate}
		\end{art}

		\begin{art}\label{funcionesComiteAcademico}
			Las funciones y atribuciones del Comité Académico son:
			\begin{enumerate}
				\item Servir de puente de información e inquietudes entre el Consejo Departamental, la Escuela y el estudiantado.
				\item Colaborar en iniciativas de integración de las diferentes disciplinas y generar instancias para la generación de sus respectivas especialidades.
				\item Revisar y evaluar las políticas académicas de la Escuela. Trabajar junto a los Consejeros Académico y de Postgrado en la formulación de propuestas de mejoramiento en términos de docencia, investigación y extensión.
				\item Velar por el cumplimiento general de la normativa de la Escuela dentro de sus diferentes organismos.
				\item Desautorizar públicamente una actuación del Comité Ejecutivo, o alguno de sus miembros; del Consejero Académico y del Consejero de Postgrado.
				\item Velar por el cumplimiento del presente Cuerpo de Estatutos, tanto el Estatuto principal como sus Apartados.
				\item Remplazar al Consejero Académico según lo estipulado en el artículo \ref{requisitosEjecutivo}.
				\item Formular la solicitud de destitución de su cargo a cualquier miembro de los organismos del CAi, por grave infracción a este estatuto o notable abandono de deberes.
			\end{enumerate}
		\end{art}

		\begin{art}\label{funcionesConsejoAcademico}
			Las funciones y atribuciones del Consejo Académico son:
			\begin{enumerate}
				\item Discutir y tomar postura frente a las políticas académicas de la Escuela o alguno de sus organismos.
				\item Pronunciarse sobre los temas académicos planteados por el Consejo Académico UC y por la Federación de Estudiantes de la Universidad Católica (FEUC) a través de su Consejo de Presidentes o cualquiera de sus instancias que considere correspondiente. En caso de tratarse de
				una votación del Consejo Académico UC, esta decisión no se considerarán los votos de los delegados de postgrado, salvo que sea una decisión que tenga relación con los estudiantes de postgrado. Esto será así mientras estos alumnos de postgrado no sean representados por la Federación. Si un tema es competencia de postgrado queda a criterio del Consejeros Académicos de Pregrado y Postgrado, la cual podría ser apelable por los delegados de postgrado.
				\item Someter a plebiscito cualquier decisión que considere pertinente, en caso de presentarse mayoría simple del total de los votos posibles.
				\item Someter a aprobación toda consulta, elección o plebiscito que no contare con el quórum mínimo para su validación.
				\item Velar por el cumplimiento del presente cuerpo de Estatutos.
				\item Destituir de su cargo a todo miembro del Consejo Académico que cumpliere con algunas de las causas establecidas como motivo de cesación de cargo en el presente estatuto, artículo \ref{requisitosEjecutivo}.
			\end{enumerate}
		\end{art}

		\begin{art}
			Las temáticas que deben ser ampliamente debatidas y aprobadas por ambos Consejos son:
			\begin{enumerate}
				\item Destitución de algún miembro del Comité Ejecutivo por grave incumplimiento de sus funciones.
				\item Modificaciones y reformas al presente estatuto de acuerdo a la parte V.
				\item Elección del Comité Ejecutivo del CAi según lo estipulado en el artículo \ref{eleccionesCAi}, inciso \ref{cierre} del presente estatuto.
			\end{enumerate}
		\end{art}

		\begin{art}
			16,7\% de los ingresos que perciba el CAi por parte del Decanato serán destinados al financiamiento de los Consejos Generacional y Académico, según la siguiente repartición:
			\begin{enumerate}
				\item 10\% destinado al Consejero Académico.
				\item 80\% a Fondos Concursables, cumpliendo lo siguiente:
					\begin{enumerate}
						\item Los únicos participantes serán los Delegados, y los proyectos deberán estar orientados a cubrir las necesidades e inquietudes del estudiantado.
						\item El fondo deberá repartirse en dos períodos, según como lo determine el Comité Ejecutivo, con el fin de cubrir ambos semestres.
						\item Las bases de los fondos concursables serán definidas por una comisión conformada por el Consejero Académico, el Tesorero del Comité Ejecutivo y un delegado correspondiente  a cada Consejo elegido en la primera sesión ordinaria del semestre.
						\item Las bases de los fondos concursables deberán estar en posesión de los delegados a más tardar un mes después a partir de la primera sesión ordinaria del semestre.
						\item El jurado evaluador de los fondos concursables estará compuesto por el Consejero Académico, un miembro del Comité Ejecutivo definido por este y un  miembro de cada Consejo elegido en la primera sesión ordinaria del semestre.
						\item Los montos ganados se pueden utilizar hasta el último día de postulación del siguiente período o hasta que los Delegados que se hayan adjudicado los fondos cesen en su cargo.
						\item El dinero no utilizado el primer período se sumará al monto destinado a fondos concursables del segundo periodo de postulación.
					\end{enumerate}
				\item 10\% al funcionamiento administrativo de los Consejos, los cuales serán repartidos en partes iguales. Este dinero deberá ser puesto a disposición del Consejero Académico y del Presidente del Comité Ejecutivo respectivamente a partir de la primera sesión ordinaria de cada Consejo.
			\end{enumerate}
		\end{art}

		\begin{art}
			Los Jefes de Comisión y sus respectivos comisionados del CAi colaborarán con el Comité Ejecutivo en el área específica de cada uno y participarán en las sesiones de dicho organismo en los casos y formas que el Presidente determine.
		\end{art}

	\section{Elecciones, requisitos y sanciones}\label{elecciones}

		\begin{art}\label{porcentajeMinimo}
			Tendrán derecho a votar todos los alumnos regulares de Pregrado de la Escuela de Ingeniería, así como los alumnos de Magíster en Ingeniería, Magíster en Ciencias de la Ingeniería y Doctorado, salvo cuando este estatuto específicamente indique algo distinto. Tratándose de la elección del Representante de College, tendrán derecho a voto los alumnos de College de Licenciatura en Ciencias Naturales que hayan cursado Desafíos de la Ingeniería, al menos 30 créditos de la malla de Ingeniería, y que al menos tengan un ramo inscrito de la malla de Ingeniería al momento de votar.

			Todas las elecciones y plebiscitos son válidos solo si en ellos vota al menos el 35\% de los alumnos con derecho a voto, salvo que en este estatuto se establezca explícitamente otra cosa. En caso de no cumplirse este quórum le corresponder a decidir al Consejo que haya llamado a plebiscitar.
		\end{art}

		\begin{art}\label{definicionEleccionesYPlebiscitos}
			Las elecciones o plebiscitos se realizarán siempre a través de una votación universal, directa, secreta, libre e informada, de la siguiente manera:
			\begin{enumerate}
				\item Puede ser tanto de manera presencial como electrónica. El encargado de decidir esto será el Consejo que haga el llamado en la misma sesión en que se inicie el proceso respectivo, según acuerdo de la mayoría absoluta de los votos posibles. En el caso de elecciones, esta sesión deberá ser el día que se cierre el proceso de inscripción. En el caso de plebiscitos, será la misma sesión en que se realice el llamado a éste.
				\item Será responsabilidad del \emph{TRICEL} organizar y salvaguardar la regularidad del proceso eleccionario o plebiscito, de acuerdo a lo estipulado en el artículo 1o del Reglamento \emph{TRICEL}.
				\item La cuenta de votos se realizará públicamente al final del último día de votación. Por esto, la(s) urna(s) se guardará(n) en el CAi bajo llave, en el lugar que estipule el TRICEL, bajo juramento de sus miembros de no hacer ni permitir manipulaciones de ningún tipo.
			\end{enumerate}
		\end{art}

		\begin{art}\label{eleccionesCAi}
			Los miembros de Comité Ejecutivo, así como el Consejero Académico y el consejero de Postgrado, durarán un año en sus funciones, pudiendo ser reelegidos si cumplieren con los requisitos. La votación de estos se realizará en una elección que se llevará a cabo de acuerdo a las siguientes normas:
			\begin{enumerate}
				\item El primer día hábil de octubre se abrirán las inscripciones para los postulantes. Se entenderá como postulante tanto las listas candidatas al Comité Ejecutivo, como los candidatos a Consejero Académico y Consejero de Postgrado. La apertura de inscripciones deberá ser avisada públicamente con una semana de anticipación. Este proceso de inscripción durará 10 días hábiles, con una preinscripción obligatoria de los candidatos al finalizar los primeros 7 días hábiles del proceso.

				Las fechas podrán ser trasladadas con un margen máximo de una semana, previa aprobación del \emph{Consejo Generacional} bajo quórum calificado.
				
				\item Las listas candidatas al Comité Ejecutivo podrán presentarse como cerradas (con los cargos previamente asignados), o abiertas (que los cargos se determinen de acuerdo a la votación obtenida por cada uno de los miembros de la lista). En el caso de presentarse una lista abierta, todos sus integrantes deberán cumplir con lo requerido por los artículos \ref{requisitosEjecutivo} y \ref{requisitosPresidente} de esta sección.
				
				\item En caso de empate entre candidatos de una lista abierta, la lista tendrá que dirimir el empate y entregarlo al CAi en ejercicio con un plazo máximo de 2 días hábiles posterior al fin del recuento de votos.
				
				\item No existe restricción con respecto al número máximo de postulantes a ninguna de las tres elecciones, siempre que todos cumplan con los requisitos estipulados en el presente estatuto.
				
				\item El número de personas inscritas en una lista debe corresponder, al menos, al número de cargos estatutarios del Comité Ejecutivo. De no cumplir esto, la lista no puede ser candidata al Centro de Alumnos.

				\item El proceso de preinscripción es obligatorio para todas las listas candidatas al Comité Ejecutivo, candidatos a Consejero de Pregrado y Consejero de Postgrado, el cual consiste en:
					\begin{enumerate}
						\item A más tardar a las 18:00 del séptimo día hábil de abierto el proceso, cada conjunto
						de postulantes a candidatos debe entregar al CAi una lista que contenga el nombre, programa de la Escuela al que pertenece, generación, promedios ponderados semestrales de los últimos tres semestres, promedio ponderado acumulado, mail UC y número de créditos aprobados de cada uno de los posibles integrantes del Comité Ejecutivo, candidatos a Consejero Académico y a Consejero de Postgrado. Se debe especificar el cargo al que postula cada candidato. 

						\item El TRICEL tiene tres días hábiles, desde cerrado el proceso de preinscripción, para validar los requisitos de todos los preinscritos.
						
						\item Tanto el Comité Ejecutivo como los Consejeros deben ser candidatos que hayan aprobado el proceso de preinscripción.
						
						\item Las listas no pueden ser publicadas, física ni electrónicamente, hasta la validación de la preinscripción por parte del TRICEL.
					\end{enumerate}

				\item Para inscribirse, se deberá entregar al CAi lo siguiente a más tardar al décimo día hábil de abierto el proceso a las 18:00 horas:
					\begin{enumerate}
						\item Para el Comité Ejecutivo:
							\begin{enumerate}
								\item La lista definitiva de candidatos, todos los cuales deben haber aprobado el proceso de preinscripción. En caso de estar previamente asignados los cargos, estos deben ser aquí especificados.
								\item\label{foto} Una foto (mín. $500 \times 500$ pixeles) de cada uno de los integrantes de la lista. 
								\item El programa o plan de trabajo de la lista para su posterior publicación.
								\item Un resumen del documento anterior con largo máximo de una carilla tamaño oficio.
								\item El nombre de su apoderado, el cual no podrá ser parte del Comité Ejecutivo del CAi en ejercicio.
							\end{enumerate}

						\item En el caso de los candidatos a Consejero Académico:
							\begin{enumerate}
								\item El nombre del candidato, el cual debe haber aprobado el proceso de preinscripción.
								\item\label{foto} Una foto (mín. $500 \times 500$ pixeles).
								\item Un plan de trabajo y propuestas para su período.
							\end{enumerate}

						\item En el caso de los candidatos a Consejero de Postgrado:
							\begin{enumerate}
								\item El nombre del candidato, el cual debe haber aprobado el proceso de preinscripción.
								\item\label{foto} Una foto (mín. $500 \times 500$ pixeles).
								\item Un plan de trabajo que detalle los lineamientos, objetivos y propuestas para su período.
							\end{enumerate}
					\end{enumerate}

				\item\label{cierre}Se cerrará el proceso de inscripción diez días hábiles después de haber sido abierto. Si finalizado este plazo alguna elección no tuviera al menos dos postulantes, dicha elección se llevará a cabo aprobando o rechazando al único postulante. Para que un postulante salga elegido bajo esta modalidad es necesario que lo apruebe más del 70\% de los votos válidamente emitidos. En caso contrario los postulantes deberán ser aprobados por ambos consejos mediante quórum calificado.
				
				\item Una vez cerrado el proceso de inscripción, si alguna de las listas debe modificar a sus candidatos sólo puede realizar un cambio entre el resto de los miembros de la misma. Esto podrán hacerlo siempre y cuando continúen cumpliendo con lo establecido en el inciso E.
				
				\item En el caso de no presentarse ninguna lista dentro de ambos períodos de inscripción, el CAi en ejercicio continuará con sus funciones hasta el mes de marzo del año siguiente, mes en el que se iniciará un nuevo proceso eleccionario. Las fechas de dicho proceso deberán ser definidas en  el primer consejo ordinario del Consejo Generacional posterior a las elecciones, siendo el último plazo para la realización de éstas el primer día de abril. El mismo proceso se seguirá en el caso de que se rechace la lista única tanto en las elecciones o en los Consejos, según lo estipulado en el artículo \ref{eleccionesCAi}.
				
				\item El Comité Ejecutivo deberá publicar en su fichero y página web los programas de las listas a lo más dos días hábiles después del cierre de inscripciones.
				
				\item El CAi deberá reproducir un resumen de los programas de las listas candidatas en número máximo de 1000 ejemplares por lista sólo si estas lo solicitaren. En caso de presentarse 3 o más listas se repartirá el total de 2000 ejemplares entre las listas postulantes. El CAi dispone de dos días hábiles para entregar a los candidatos las copias de los resúmenes de sus respectivos programas. El presente artículo también rige para los candidatos a Consejero Académico y Consejero de Postgrado. Serán los candidatos, en ambos casos, los encargados de distribuirlos.
				
				\item A lo más 6 días hábiles después del cierre de las inscripciones deberá efectuarse una asamblea general de presentación de los postulantes a las tres elecciones, cuyo objetivo será dar a conocer sus programas y responder las inquietudes del alumnado.
				
				\item \label{periodo}A lo más dos días hábiles después de la asamblea de presentación de las listas, se iniciará el período de votación, que se prolongará por dos días hábiles.
				
				\item \label{votantes}Los alumnos que se encuentren cursando Pregrado en la Escuela de Ingeniería tendrán derecho a votar por Comité Ejecutivo y por Consejero Académico. Aquellos que cursen Postgrado en la misma tenderán derecho a votar por Comité Ejecutivo y por Consejero de Postgrado.
				
				\item Los alumnos que cursen de forma simultánea Pregrado y Postgrado en la Escuela de Ingeniería podrán sufragar en una sola oportunidad, votando de forma simultánea al Comité Ejecutivo, Consejero Académico y Consejero de Postgrado.
				
				\item El resultado de las elecciones se determinará de la siguiente forma:
					\begin{enumerate}
						\item Si en una elección hay un solo postulante, se rige según el inciso \ref{cierre} del presente artículo.
						\item Si hay sólo dos postulantes resultará elegido aquél que obtenga simple mayoría sobre el otro. Para estos efectos se entenderá que existe simple mayoría si una de las dos listas tiene al menos un voto más que la otra. Para estos efectos no se tomarán en cuenta los votos blancos ni nulos. Si ambos postulantes obtienen el mismo número de votos, le corresponderá decidir a los consejos Generacional y Académico por mutuo acuerdo.
						\item Si hay tres o más postulantes, los que obtengan las dos primeras mayorías en la votación, irán a segunda vuelta, siempre y cuando ninguno de ellos obtenga más del 50\% de los votos válidamente emitidos. Se entenderá como válidamente emitidos los votos totales sin contar blancos ni nulos. La segunda vuelta se realizará a lo más 5 días hábiles después del término de la primera vuelta, y se regirá como una votación de sólo dos postulantes.
					\end{enumerate}
			\end{enumerate}
		\end{art}

		\begin{art}\label{requisitosEjecutivo}
			Los postulantes al Comité Ejecutivo deberán cumplir, al momento de la preinscripción:
			\begin{enumerate}
				\item Ser alumnos regulares de la Escuela de Ingeniería de la Pontificia Universidad Católica de Chile.
				\item En el caso de alumnos de pregrado:
					\begin{enumerate}
						\item Tener a lo menos tres semestres cursados en la unidad académica o 150 créditos aprobados de algún currículum de Ingeniería Civil al momento de la elección.
						
						\item Tener un promedio ponderado de los últimos tres Promedio Semestral de notas (PPS), al momento de la elección, superior o igual a 4,50.

						Esta norma tiene por objetivo que ningún miembro del Comité Ejecutivo caiga en causal de eliminación durante su ejercicio. El organismo oficial de validación es la Dirección de Pregrado de la Escuela de Ingeniería.
					\end{enumerate}
				\item En el caso de alumnos de postgrado:
					\begin{enumerate}
						\item En el caso de alumnos en primer semestre del programa, cumplir con los requisitos establecidos para alumnos de pregrado.
						\item En el caso de alumnos de segundo semestre en adelante, contar con un PPA superior o igual a 5,00.

						Esta norma tiene por objetivo que ningún miembro del Comité Ejecutivo caiga en causal de eliminación durante su ejercicio. El organismo oficial de validación es la Dirección de Postgrado de la Escuela de Ingeniería.
					\end{enumerate}

				\item No haber caído en causal de eliminación durante el año previo al de la elección.
				
				\item Tener antecedentes de probidad académica intachables. 

				El TRICEL será responsable de velar por todos los incisos de este artículo.
			\end{enumerate}
		\end{art}

		\begin{art}\label{requisitosPresidente}
			El candidato a Presidente del CAi deberá tener un promedio ponderado acumulado de notas (PPA) superior o igual 4,80 y a lo menos 200 créditos aprobados de algún programa de la Escuela de Ingeniería Civil al momento de la preinscripción.
		\end{art}

		\begin{art}\label{requisitosJefeInvestigacion}
			El candidato al cargo de Jefe de Investigación deberá específicamente ser alumno regular o en vías de grado de Postgrado.
		\end{art}

		\begin{art}\label{requisitosCAPregrado}
			Los candidatos a Consejero Académico deberán cumplir con los siguientes requisitos, al momento de la preinscripción:
			\begin{enumerate}
				\item Ser alumno regular de algún major o título profesional de Ingeniería Civil con PPA superior 4.80.
				\item Haber aprobado al menos 200 créditos al momento de inscribir la candidatura.
				\item No haber caído en causal de eliminación.
				\item Tener antecedentes de probidad académica intachables. 


				El TRICEL será responsable de velar por todos los incisos de este artículo.
			\end{enumerate}
		\end{art}

		\begin{art}\label{requisitosCAPostegrado}
			Los candidatos al cargo de Consejero de Postgrado deberán cumplir con los siguientes requisitos, al momento de la preinscripción:
			\begin{enumerate}
				\item Ser alumno regular o en vías de grado de Postgrado con PPA superior 5.00.
				\item Haber completado al menos un semestre en algún programa de Postgrado de la Escuela de Ingeniería.
				\item No haber caído en causal de eliminación.
				\item Tener antecedentes de probidad académica intachables. 

				El TRICEL será responsable de velar por todos los incisos de este artículo.
			\end{enumerate}
		\end{art}

		\begin{art}\label{eleccionExtraordinariaCAPregrado}
			En caso de no presentarse ningún candidato al cargo de Consejero Académico después de los dos períodos estipulados en el artículo \aaref{eleccionesCAi}{cierre} de esta parte, el Consejo Académico elegirá de entre sus miembros (que cumplan con los requisitos de Consejero Académico) a alguien que asuma el cargo. Esta obligación recaerá ineludiblemente sobre alguno de los Delegados, y ellos deben estar conscientes de dicha condición antes de ser elegidos como tales. El Delegado que asuma el cargo tendrá la opción de designar a un reemplazante para su labor previa como representante de especialidad. Todas estas obligaciones corresponderán al Consejo Académico en ejercicio.
		\end{art}

		\begin{art}\label{nominacionSubconsejeros}
			La nominación de cada subconsejero se realizará posterior a la elección de Comité Ejecutivo y Consejeros Académicos, bajo los siguientes criterios:
			\begin{enumerate}
				\item Para el caso del Subconsejero Académico, será el Jefe de Docencia, a menos que el Consejero Académico proponga una persona distinta, quien debe cumplir los requisitos del artículo \ref{requisitosEjecutivo}
				
				\item Para el caso del Subconsejero de Postgrado, este tiene que pertenecer al programa de Postgrado al que el Consejero de Postgrado no pertenece. Podría ser el Jefe de Investigación, a menos que el Consejero de Postgrado proponga a una persona distinta, quien debe cumplir con el requisito planteado anteriormente y en los artículos \ref{requisitosEjecutivo} y \ref{requisitosJefeInvestigacion}. 
				
				\item Ambas propuestas, deben ser validadas por mayoría simple en el Consejo Académico en el primer consejo ordinario de la directiva electa.
				
				\item De no ser validado, cada Consejero puede proponer otro nombre entre los delegados existentes que cumplan los requisitos de los artículos \ref{requisitosEjecutivo} y de ser necesario el artículo \ref{requisitosJefeInvestigacion}. 
			\end{enumerate}
		\end{art}

		\begin{art}\label{definicionesDelegados}
			Los Delegados y el Representante de College durarán un año en sus funciones. Esta elección se hará de acuerdo a las siguientes normas:
			\begin{enumerate}
				\item Para alumnos cuyo año de ingreso sea menor al de la tercera generación, cada estudiante tendrá derecho a votar por dos candidatos de su respectiva generación. En ningún caso podrá acumular estos dos votos en un mismo candidato.
				\item Para alumnos cursando un major y pertenecientes al menos a la tercera generación, cada estudiante tendrá derecho a votar por un candidato de su respectivo major y por dos candidatos de su respectiva generación. En ningún caso podrá acumular estos dos votos en un mismo candidato.
				\item Para alumnos cursando una especialidad, cada alumno tendrá derecho a votar por un candidato de su departamento, especialidad o centro, además de poder votar por un candidato del Departamento de Ingeniería Industrial en caso que este cursando una mención asociada a ese departamento. También deberá votar por dos candidatos de su respectiva generación. En ningún caso podrá acumular estos dos votos en un mismo candidato.
				\item Para los alumnos de la quinta generación que se encuentren cursando su último semestre de major e iniciando su especialidad, cada alumno tendrá la opción de decidir si votar por un delegado de su major o de su departamento, especialidad o centro. En caso de elegir votar por especialidad, y en caso de que el alumno este cursando una especialidad Industrial, también podrá votar por un candidato del Departamento de Ingeniería Industrial.
				\item Para alumnos cursando algún programa de Postgrado, cada alumno tendrá derecho a votar por un candidato de su departamento, especialidad o centro. También deberá votar por dos candidatos a delegado generacional por postgrado. En ningún caso podrá acumular estos dos votos en un mismo candidato.
				\item Los alumnos que cursen de forma simultánea Pregrado y Postgrado en la Escuela de Ingeniería podrán sufragar en una sola oportunidad. Cada alumno tendrá derecho a votar por un candidato de su departamento, especialidad o centro, además de poder votar por un candidato del Departamento de Ingeniería Industrial en caso que este cursando una mención asociada a ese departamento. También deberá votar por dos candidatos a delegado generacional, los cuales podrá repartir entre los candidatos de su respectiva generación o de Postgrado. En ningún caso podrá acumular estos dos votos en un mismo candidato.
				\item Para alumnos de College de Licenciatura en Ciencias Naturales que cumplan los requisitos del artículo \ref{porcentajeMinimo}, cada alumno tendrá derecho a votar por un candidato a Representante de College.
				\item Durante la cuarta semana a partir del inicio del primer periodo académico para alumnos antiguos se abrirán las inscripciones para los postulantes. La apertura de inscripciones deberá ser avisada con una semana de anticipación. No existe restricción con respecto al número máximo de postulantes, siempre que todos cumplan con los requisitos estipulados en el presente estatuto. Las fechas podrán ser adelantadas o retrasadas con un margen máximo de dos semanas, previa aprobación del Consejo respectivo bajo quórum calificado.
				\item Se cerrará el proceso de inscripciones cinco días hábiles después de haber sido abierto. Durante estos días, los candidatos tendrán la obligación de entregar al CAi los siguientes documentos:
					\begin{enumerate}
						\item Un escrito que contenga el nombre, mail, PPA (puntaje de ingreso en caso de alumnos de primer año), generación y especialidad, major o generación por la que postula.  En el caso del Representante de College, deberá especificar que pertenece a dicho programa.
						\item Una foto (mín. $500 \times 500$ pixeles) del candidato.
					\end{enumerate}
				Esta misma información deberá ser enviada por correo electrónico al Secretario General del Comité Ejecutivo.
				\item Si sólo se hubiera presentado un candidato o ninguno para alguna generación, departamento, major, especialidad que no esté representada por un departamento o Representante de College, se reabrirán las inscripciones para todos los cargos durante los próximos cinco días hábiles para dar la posibilidad que se inscriban más candidatos.
				\item A lo más cuatro días hábiles después del cierre de las inscripciones se iniciará el periodo de votación, el cual se prolongará por dos días hábiles.
				\item El resultado de las elecciones se determinará de la siguiente forma:
					\begin{enumerate}
						\item En el caso de los delegados Generacionales serán elegidas las tres primeras mayorías. Para los Delegados Académicos será elegida la primera mayoría de cada major, departamento y especialidad no representada por un departamento.
						\\
						Tratándose del Representante de College será elegida la primera mayoría.
						\item En caso de no llenarse los cupos para los distintos cargos, se perderá la representación correspondiente a los cargos no ocupados.
						\item En caso de empate saldrá elegido el de mayor prioridad académica y en el caso de los novatos, el que haya entrado con mayor puntaje.
					\end{enumerate}
			\end{enumerate}
		\end{art}

		\begin{art}\label{requisitosDelegados}
			Para postular al cargo de Delegado se debe cumplir con los siguientes requisitos:
			\begin{enumerate}
				\item Ser alumno regular de Pregrado, de Magíster en Ciencias de la Ingeniería, de Magíster en Ingeniería o Doctorado de la Escuela de Ingeniería.
				\item Para los Delegados Generacionales:
					\begin{enumerate}
						\item Alumnos de pregrado: pertenecer a la generación que representaría y estar cursando a lo menos 30 créditos del currículum. Además, presentar en la oficina del CAi 50 firmas que validen la candidatura.
						\item Alumnos de postgrado: estar trabajando bajo tutela de un profesor en una tesis de Magister o Doctorado.
					\end{enumerate}
				\item Para los Delegados Académicos; en el caso de major, estar inscrito en el major al cual se postula como representante y estar cursando al menos la tercera generación; en el caso de los departamentos o especialidades sin departamento, estar cursando la especialidad o mención a la cual se postula como representante y contar con el grado de licenciado o estar a 30 créditos o menos de cumplirlo. Además el postulante, en el caso de ser alumno de pregrado, debe estar cursando a lo menos 30 créditos del currículum, y llevar realizados al menos 40 créditos o estar cursando al menos un ramo de la especialidad, major o mención que declara cursar. En el caso de los alumnos de postgrado, deben estar trabajando bajo tutela de un profesor en una tesis del departamento o especialidad a la cual postula.
			\end{enumerate}
		\end{art}

		\begin{art}\label{requisitosRepresentanteCollege}
			Para postular al cargo de Representante de College se debe cumplir con los siguientes requisitos:
			\begin{enumerate}
				\item Ser alumno regular UC y poseedor de la Licenciatura de Ciencias Naturales y Matemáticas con un major en Ingeniería. 
				\item O bien, cumplir con los siguientes requisitos:
					\begin{enumerate}
						\item Ser alumno regular del programa College de Licenciatura en Ciencias Naturales.
						\item Encontrarse cursando como mínimo el quinto semestre en el programa.
						\item Haber aprobado al menos 80 créditos de la malla de ingeniería, excluyendo OFG’s a excepción de Habilidades Comunicativas para Ingenieros y Ética para Ingenieros. 
						\item Estar cursando a lo menos 30 créditos del currículum. 
					\end{enumerate}
			\end{enumerate}
		\end{art}

		\begin{art}\label{ceseCargo}
			Los miembros tanto del Consejo Académico como Generacional cesarán en sus cargos, para todos sus efectos, por alguna de las siguientes razones:
			\begin{enumerate}
				\item Término de su periodo.
				\item\label{alumno_regular} Tratándose de los delegados, dejar de ser alumno regular de la unidad académica.
				\item\label{sin_ramos} Tratándose de los delegados, no cursar ramos del currículum o dejar de trabajar bajo tutela de un profesor en una tesis de postgrado en un semestre dentro de su ejercicio.
				\item\label{destitucion} Ser destituido por el Consejo al que pertenece por grave infracción a este estatuto.
				\item\label{renuncia} Renuncia aceptada por el Presidente o formulada como indeclinable. Se deberá presentar la renuncia de forma escrita.
				\item Tratándose del Presidente, por renuncia aceptada por ambos Consejos o formulada como indeclinable.
				\item\label{inhabilitacion} Por ser inhabilitado en razón de su incapacidad como dirigente estudiantil. Para entender acogida la inhabilidad se requerirá lo siguiente:
					\begin{itemize}
						\item La aprobación de cada Consejo, la cual se obtendrá mediante la aprobación de dos tercios de los votos posibles, si se tratase de un miembro del Comité Ejecutivo, del Consejero Académico, o del Consejero de Postgrado.
						\item La aprobación de dos tercios de los votos posibles del Consejo al que pertenezca excluyéndose al afectado, en caso de tratarse de algún delegado o del Representante de College.
					\end{itemize}
				\item\label{causal} Caer en causal de eliminación.
				\item Involucrarse en situaciones de probidad académica reprochables. El Consejo al que pertenezca será responsable de velar por esto.
				\item Asumir otro cargo de representación que integre el Consejo Generacional o Académico del CAi.
				\item Ejercer alguno de los cargos de representación enunciados a continuación:
					\begin{itemize}
						\item Directiva de la FEUC.
						\item Consejero Superior FEUC.
					\end{itemize}
				\item Tratándose de un miembro del Comité Ejecutivo, ejercer como \emph{Consejero Territorial FEUC}.
				\item Ser elegido o ejercer algún cargo de representación estudiantil en otra Unidad Académica.
				\item Tratándose del Representante de College, dejar de ser alumno regular de su unidad académica o de la Escuela de Ingeniería, asumir otro cargo de representación estudiantil o no cursar ramos del currículum.
			\end{enumerate}
		\end{art}

		\begin{art}\label{maxAusencias}
			El número máximo de ausencias a las sesiones del Consejo correspondiente aceptadas a los Delegados, al Representante de College y a los representantes del Comité Ejecutivo es como sigue:
			\begin{enumerate}
				\item \label{ausencias_A} Sesiones Ordinarias: Dos ausencias si no son justificadas, o tres ausencias si alguna de ellas es debidamente justificada.

				Para que una ausencia se considere justificada se debe: avisar con al menos 5 horas de anticipación del consejo y tener algún compromiso previamente establecido a la hora del consejo, o un evento de fuerza mayor que imposibilite asistir al consejo. 
				Quien tiene el poder de validar una justificación es el Secretario General del Comité Ejecutivo, después de recibir el documento oficial de justificación. Es el Secretario quien define cómo indicar cada razón de justificación en el acta.

				\item \label{ausencias_B} Sesiones Extraordinarias: un máximo de tres ausencias cada diez sesiones de consejos extraordinarios, siempre y cuando sean citados con al menos 36 horas de anticipación. La cantidad de ausencias se extiende en tres más para cada conjunto de diez Consejos Extraordinarios. \\
				En el caso de los delegados de postgrado, las ausencias a aquellos consejos extraordinarios para decidir alguna votación ponderada de la FEUC no será considerada.

			\end{enumerate}

			El tiempo mínimo que debe estar un miembro del Consejo durante una sesión para que su asistencia sea considerada es de 30 minutos. La asistencia de los miembros del Consejo debe ser publicada en la página web del CAi.\\
			En caso de que las restricciones A o B del presente artículo sean violadas, el Delegado será sancionado con la pérdida inmediata del derecho a voz y voto en el Consejo al que pertenece, si bien podrá seguir representando a los alumnos frente a los profesores de su generación, major, departamento o especialidad, según sea el caso. En el caso del Comité Ejecutivo, este perderá el derecho a voz y voto por el resto de su período.\\
			Tratándose de los delegados generacionales de la primera y segunda generación, se considerará como ausencia al Consejo Académico el hecho de que ninguno de los tres esté resente. En el caso de no cumplir con las restricciones A o B del presente artículo, se considerará que pierden su derecho a voz y voto por el resto de su período en dicho Consejo.
		\end{art}

		\begin{art}\label{ceseCargoEjecutivo}
			Si alguno de los miembros del Comité Ejecutivo, exceptuando al Presidente, cesare en su cargo, será reemplazado por quien designe el Presidente, con el posible veto de alguno de los Consejos, obtenido mediante votación por quórum calificado.
		\end{art}

		\begin{art}\label{ceseCargoPresidente}
			Si fuera el Presidente el que cesare en su cargo, asumirá el Primer Vicepresidente por lo que resta del período, efectuando una designación del nuevo Primer Vicepresidente según lo establecido en el artículo anterior.
		\end{art}

		\begin{art}\label{reemplazoDelegados}
			Si un delegado cesare en su cargo, se designará a aquel candidato que hubiere obtenido la siguiente mayoría no electa. En caso de que no existiese o no aceptara, el delegado saliente tendrá la posibilidad de proponer un reemplazante al Consejo al que pertenezca, que deberá ser aprobado por mayoría simple. De ser rechazado, el puesto quedará vacante. Si el delegado que cesa en su cargo hubiere perdido el derecho a voto durante su periodo, el delegado que lo reemplazare carecerá del mismo por toda la extensión de su periodo. Si aún lo mantiene, mantendrá el número de ausencias del delegado saliente.
		\end{art}

		\begin{art}
			Si el Representante de College cesare en su cargo, se designará a aquel candidato que hubiese obtenido la siguiente mayoría no electa. En caso de que no existiese o no aceptara, el Representante saliente tendrá la posibilidad de proponer un remplazante al Consejo Académico, el que deberá cumplir los requisitos del artículo \ref{requisitosRepresentanteCollege} y ser aprobado por mayoría simple en el Consejo Académico. De ser rechazado el puesto quedará vacante. Si el representante que cesa en su cargo hubiere perdido el derecho a voto durante su periodo, el representante que lo reemplazare carecerá del mismo por toda la extensión de su período Si aún lo mantiene, mantendrá el número de ausencias del Representante saliente.
		\end{art}

		\begin{art}\label{reemplazoCA}
			Si el Consejero Académico cesare en su cargo, su reemplazante será el Subconsejero Académico.
		\end{art}

		\begin{art} \label{eleccionExtraordinariaCAPostgrado}
			En caso de que el cargo de Consejero de Postgrado quedase vacante, debe procederse como se indica a continuación:
			\begin{enumerate}
				\item\label{cp_renuncia} En caso de quedar vacante por renuncia, cesación anticipada del cargo o deposición de un Consejero en ejercicio, su reemplazante será el Subconsejero de Postgrado.
				
				\item Si no se hubiese presentado ningún candidato al cargo de Consejero de Postgrado después de los dos períodos estipulados en el artículo \aaref{eleccionesCAi}{cierre} de esta parte, o no hubiese una mayoría no electa a quién ofrecer el cargo de acuerdo a lo establecido en la letra A de este artículo, el Consejo Académico deberá solicitar al presidente del Comité Ejecutivo la elaboración de una lista, jerarquizada según orden de elegibilidad, de tres candidatos que cumplan con los requisitos establecidos para postular al cargo de Consejero de Postgrado. El Consejo Académico someterá a aprobación a los candidatos, de acuerdo al orden propuesto. Se ofrecerá el cargo a quién obtuviese mayoría de los votos presentes. Si a quién se le ofreciere el cargo no aceptase, se repetirá el proceso con el siguiente candidato. De no existir un candidato a quién ofrecer el cargo, o de haber transcurrido 3 semanas desde la renuncia o cesación del cargo del Consejero anterior, se declarará desierto el cargo por lo que reste del periodo.
			\end{enumerate}
		\end{art}

	\section{Sesiones}\label{sesiones}

		\subsection*{Sobre las sesiones del Comité Ejecutivo}

			\begin{art} \label{sesionesEjecutivo}
				El Comité Ejecutivo podrá sesionar cuando lo estime conveniente, previa notificación a todos sus miembros. Estas sesiones serán presididas por el miembro del Comité Ejecutivo de mayor jerarquía entre los presentes. Para esto se tendrá en cuenta la jerarquía establecida en el artículo  \aaref{organismosCAi}{jerarquia}.
			\end{art}

		\subsection*{Sobre las sesiones de los Consejos}
			\addcontentsline{toc}{subsection}{Sobre las sesiones de los Consejos}

			\begin{art}\label{consejos}
				Cada Consejo debe sesionar en forma ordinaria una vez al mes durante el año académico y de forma pública. Se entenderá el año académico como el periodo de marzo a diciembre. El Generacional sesionará bajo la presidencia del Presidente del Comité Ejecutivo del Centro de Alumnos y el Académico será presidido por el Consejero Académico.
			\end{art}

			\begin{art}\label{consejoOrdinario}
				La citación a sesiones ordinarias de cada Consejo será notificada con un mínimo de dos días hábiles de anticipación. Junto con la citación se incluirá un borrador de la tabla con los temas a tratar. La citación será notificada a cada miembro del Consejo vía correo electrónico, además de ser publicada en la página web del CAi.
			\end{art}

			\begin{art}\label{consejoExtraordinario}
				Cada Consejo puede ser citado en forma extraordinaria por el Comité Ejecutivo, por un tercio de los Delegados o quién lo presida, debiendo ser notificado a todos los miembros del consejo vía correo electrónico con al menos 24 hrs de anticipación.
			\end{art}

			\begin{art}\label{facultadesPrecide}
				La facultad de quién presida la sesión comprende:
				\begin{enumerate}
					\item La de dirigir el debate con amplias facultades, salvo las expresas limitaciones consagradas en este estatuto.
					\item Amonestar a cualquier asistente que perturbare el normal desarrollo de la sesión. Si el amonestado persistiere en su actitud,  quien presida el Consejo podrá expulsarlo de la sala por el resto de la sesión, previa aprobación del Consejo.
					\item Ordenar que abandonen la sala todos los asistentes que no fueren Consejeros cuando lo estimare indispensable para el normal desenvolvimiento de la sesión, previa aprobación del Consejo.
				\end{enumerate}
			\end{art}

			\begin{art}\label{quorumConsejo}
				El quórum para llevar a cabo un Consejo debe ser más del 50\% de los Delegados que no hayan perdido su derecho a voto. Además será necesaria la presencia del Consejero Académico y Jefe de Docencia o Subconsejero Académico, en el caso del Consejo Académico, y de algún miembro del Comité Ejecutivo y del Consejero Académico, o a quien este último haya otorgado poder, en el caso del Consejo Generacional. Pasados los treinta minutos de la hora fijada para el comienzo de la sesión, el quórum será de una tercera parte de los Delegados que no hayan perdido su derecho a voto, mas Consejero Académico y Jefe de Docencia o Subconsejero Académico y de algún miembro del Comité Ejecutivo y del Consejero Académico, o a quien este último haya otorgado poder, en el caso del Consejo Generacional. No ocurriendo esto, se declarará fracasada la sesión a petición de cualquier miembro del Consejo. Para el cálculo de este quórum no se contabilizarán a los delegados de postgrado si ellos no participan en la sesión del Consejo Académico.
			\end{art}

			\begin{art}\label{duracionConsejosOrdinarios}
				Las sesiones ordinarias del Consejo durarán un módulo académico (una hora y veinte minutos), extensible a un máximo de dos módulos a través de la siguiente manera:
				\begin{itemize}
					\item Diez minutos antes del término del primer módulo, los miembros del Consejo que estén presentes y tengan derecho a voto, mediante votación simple decidirán si se extiende al segundo módulo o se cita a sesión extraordinaria para continuar la discusión.
					\item En caso de continuar la discusión en un segundo módulo y ésta no finalizara, 15 minutos antes del fin de este módulo se votará, de la misma manera que antes, para decidir si se continúa la discusión en una sesión extraordinaria. Lo dispuesto en los artículos precedentes debe entenderse sin perjuicio del derecho del presidente a inscribir a los que deseen hacer uso de la palabra, limitando el tiempo de los oradores por acuerdo conjunto entre él y el Consejo.
				\end{itemize}
			\end{art}

			\begin{art}\label{cuentasPublicasOrdinarias}
				La tabla de las sesiones ordinarias de cada Consejo deberá contemplar ---al menos--- los siguientes puntos:
				\begin{enumerate}
					\item Cuenta del Comité Ejecutivo
					\item Cuenta del Consejero Académico
				\end{enumerate}
				Esto sin perjuicio a la facultad de los miembros del Consejo de proponer otros temas para la sesión, de acuerdo a lo establecido en el presente cuerpo de estatutos.
			\end{art}

			\begin{art}\label{definicionTabla}
				La tabla de cada sesión del Consejo se considerará cerrada pasados los primeros quince minutos de iniciada la sesión. Durante este tiempo, así como previo al inicio de la sesión, podrán proponer temas a la tabla todos aquellos miembros del Consejo que tengan derecho a voz, y su incorporación quedará sujeta a aprobación de quien presida la sesión. No obstante lo anterior, una solicitud que sea apoyada por un tercio de los miembros con derecho a voto presentes deberá ser incorporada a la tabla sin derecho a veto posible.
			\end{art}

			\begin{art}\label{temaPropuestoAlumnado}
				Si previo al inicio de una sesión de Consejo o durante los primeros 15 minutos de ésta, el Consejo fuese requerido a discutir, pronunciarse y/o votar en torno a un tema en particular, y dicha solicitud viniese respaldada por la firma del 5\% de los alumnos representados por el CAi, esta solicitud deberá ser acogida e incorporada a la tabla.
			\end{art}

			\begin{art}\label{fijarAsistencias}
				En la primera sesión ordinaria de cada Consejo:
				\begin{enumerate}
					\item Se solicitará la disponibilidad horaria de cada miembro del Consejo. Esta información será recopilada, de la forma que se estime conveniente, por el Secretario de cada consejo, quien intentará definir a partir de ésta el día y hora de las sesiones ordinarias que permitan conseguir la asistencia de la mayoría de los miembros del Consejo.
					
					\item Para el caso del Consejo Generacional se elegirá a un Secretario Ejecutivo, quién ocupará dicho cargo durante un semestre. Este tendrá el rol de apoyar al Secretario General en el funcionamiento del Consejo, llámese apoyo:
						\begin{enumerate}
							\item Levantamiento de actas, o actas resumen, en caso de Consejos grabados.
							\item Coordinación y publicación de asistencias del Consejo.
							\item Coordinación y manejo de comisiones.
							\item Publicación de Cuentas Públicas de Comisiones y Delegados en la página del CAi.
							\item Difusión de información dentro del Consejo Generacional.
						\end{enumerate}
					
					\item En la primera sesión ordinaria del Consejo Generacional se definirá la comisión de Fiscalización y Estatutos. Dicha comisión presentarán una cuenta pública 3 veces al año (una vez durante el primer semestre y dos veces durante el segundo). Deberá tener, por lo menos, cinco delegados generacionales y un miembro del CAi. También podrán sumarse delegados académicos a esta comisión. La comisión será liderada por dos miembros de esta.
				\end{enumerate}
			\end{art}

			\begin{art}\label{cuentasPublicasSemestrales}
				Al menos una vez al semestre, el Comité Ejecutivo, el Consejero Académico y el Consejero de Postgrado deberán rendir una cuenta de las labores efectuadas en el transcurso de su período.
			\end{art}

			\begin{art}\label{presupuestoEjecutivo}
				En la sesión ordinaria de cada Consejo correspondiente al mes de marzo, el jefe de finanzas del CAi deberá presentar un presupuesto estimado para su gestión, que explicite los ingresos y egresos esperados para el año académico. Este presupuesto debe presentar, al menos, los gastos e ingresos estimados para cada Comisión, así como de todo proyecto o actividad cuyo presupuesto supere 20 UF. Este presupuesto no debe considerarse vinculante.
			\end{art}

			\begin{art}\label{votacioncesConsejo}
				Toda votación se realizará de forma oral, salvo que quién presida la sesión o la tercera parte del Consejo determine que se haga en forma escrita, nominal y simultánea. Para efectos de votación, se considerará lo siguiente:
				\begin{enumerate}
					\item Tratándose del Consejo Generacional:
						\begin{itemize}
							\item Cada Delegado de generación tendrá un voto.
							\item El Comité Ejecutivo tendrá tres votos, correspondientes al Presidente y Vice-presidentes.
							\item El delegado de postgrado que se encuentre representado en dicha
							oportunidad tendrá un voto, con excepción de aquellas votaciones que deban ser llevadas al Consejo de Federación y no tengan relación directa con los programas de postgrado.
							\item El Consejero Académico tendrá un voto.
							\item El Consejero de Postgrado tendrá un voto.
						\end{itemize}
					\item Tratándose del Consejo Académico:
						\begin{itemize}
							\item Cada delegado de major, especialidad o centro tendrá un voto.
							\item El delegado de la primera y la segunda generación que se encuentre representando en dicha oportunidad a su generación tendrá un voto.
							\item El Jefe de Docencia del Comité Ejecutivo tendrá un voto.
							\item El Consejero Académico tendrá un voto.
							\item El Consejero de Postgrado tendrá un voto.
							\item El Representante de College tendrá un voto.
							\item El Subconsejero Académico tendrá un voto en caso que el Jefe de Docencia no esté presente.
						\end{itemize}
					\item Para evitar la sobrerrepresentación de algún representante, en el caso de una votación que requiera la aprobación de ambos consejos: los delegados que tienen presencia en ambos Consejos sólo podrán votar en el Consejo al que pertenecen; los Consejeros Académicos de pre y postgrado sólo votarán en el Consejo Académico; y el Comité Ejecutivo contará solo con los votos en el Consejo Generacional.
				\end{enumerate}
			\end{art}

			\begin{art}\label{quorumMinimoVotacion}
				El quórum mínimo para realizar cualquier votación será el 50\% de los votos posibles.
			\end{art}

			\begin{art}\label{esquemaVotacion}
				Toda moción sometida a votación se aprobará por simple mayoría de los votos presentes, salvo que en este Estatuto se especifique lo contrario. En caso de empate, la decisión final recae sobre quién presida la sesión.
			\end{art}

			\begin{art}\label{quorumCalificado}
				Las votaciones que requieran quórum calificado, serán aprobadas por dos tercios de los votos presentes.
			\end{art}

			\begin{art}\label{votacionMovilizacionParo}
				Las votaciones para decidir la participación en paros o actividades que alteren el funcionamiento normal de la Escuela deberán ser tomadas en el Consejo Generacional y requerirán quórum calificado de la forma en que se especifica en este Estatuto. En caso de decidir la participación en movilizaciones estudiantiles, estas requerirán mayoría simple. En ambos casos, para el cálculo de este quórum no se contabilizarán a los delegados de postgrado si ellos no participan en la votación.  
			\end{art}

			\begin{art}\label{votacionExterna}
				En los casos en que los resultados de votaciones deban ser llevados ante organismos externos al CAi, dicha votación deberá ser llevada a cabo por el Consejo Generacional. Ésta se presentará en la misma proporción original, redondeando al entero más cercano. Si por este sistema se produjese un empate, se decidirá por la postura más votada de entre las empatadas. Si el empate persistiese corresponderá al presidente dirimir. En el caso de ser al Consejo de Presidentes de la FEUC, o cualquier otra instancia en la que el Comité Ejecutivo tenga derecho a voto propio, solo serán válidos los votos de los Delegados.
			\end{art}

			\begin{art}\label{definicionPoder}
				Cualquier miembro del Consejo podrá delegar en otro alumno regular de la Escuela de Ingeniería su voz y su voto para una sesión determinada de dicho consejo mediante un poder escrito. Este poder deberá ser entregado al Secretario antes del inicio de la sesión correspondiente y será considerado como asistencia. En caso de que algún Consejero o Delegado deba hacer abandono de la sesión, posterior a los 30 minutos necesarios para hacer válida la asistencia, éste podrá sustituir su presencia mediante un poder escrito, denominado Poder de Reemplazo. Para efecto de quórum, los poderes serán considerados como asistencia sólo si está presente la persona a quien ha sido otorgado.
			\end{art}

			\begin{art}\label{maxPoderesRecividos}
				Ningún alumno podrá acumular más de un poder a su nombre para una misma sesión. En caso de presentarse dos o más a nombre de una misma persona, sólo será válido el que primero haya aceptado en el curso de la sesión.
			\end{art}

			\begin{art}\label{maxPoderesEmitidos}
				Un Delegado no podrá presentar un poder más de dos veces en su período para las sesiones ordinarias. En caso de votación en las sesiones extraordinarias, los delegados podrán, adicionalmente a lo anterior, dejar a lo más dos poderes simples durante su período. Esta última cifra podrá extenderse hasta un máximo de cuatro por acuerdo del Consejo si durante el periodo de los delegados en ejercicio se hubiese celebrado más de diez Consejos Extraordinarios. En el caso de los poderes de reemplazo, según el artículo \ref{definicionPoder}, los delegados poseen un máximo de dos poderes para sesiones ordinarios y dos para las sesiones extraordinarias, independientes de los poderes que envían cuando no asisten a las sesiones.
			\end{art}

		\subsection*{Sobre las sesiones del Consejo Generacional}
			\addcontentsline{toc}{subsection}{Sobre las sesiones del Consejo Generacional}

			\begin{art}
				Los Consejeros Territoriales ---por la función de representantes del alumnado de Ingeniería ante el consejo ejecutivo y de federación que el estatuto FEUC les confiere--- son reconocidos como participantes del Consejo Generacional sin derecho a voto, por lo cual, no se contará su asistencia para el quórum. Se les reconocen las siguientes facultades y derechos:
				\begin{enumerate}
					\item Derecho a voz en las sesiones ordinarias y extraordinarias del Consejo Generacional.
					\item Citar al Consejo Generacional de forma extraordinaria, previo acuerdo de la mitad o más de los Consejeros Territoriales y respetando las especificaciones establecidas en este estatuto para ello.
					\item Proponer temas en la tabla del Consejo Generacional.
				\end{enumerate}
			\end{art}

		\subsection*{Sobre las sesiones del Consejo Académico}

			\addcontentsline{toc}{subsection}{Sobre las sesiones del Consejo Académico}

			\begin{art}
				Para el correcto funcionamiento del Consejo Académico se elegirá un Secretario Académico de entre los delegados. Será un cargo de duración semestral y será elegido por mayoría absoluta de los miembros del Consejo durante la primera sesión del Consejo  y en el Consejo Ordinario de agosto.
				\\
				Si quien fue nombrado Secretario se ausentare a una sesión, se elegirá otro delegado que tomará acta en su lugar y deberá hacérsela llegar dentro de dos días hábiles.
				\\
				En caso de presentar renuncia, deberá proponer a un miembro del Consejo que lo suceda para el resto del período, sin interferir con que existan más postulantes.
				\\
				Es responsabilidad del Secretario Académico tomar asistencia y acta sobre las acciones que decida el Consejo, así como realizar la citación a los consejos ordinarios y a los extraordinarios que se consideren pertinentes.
				\\
				El acta de cada sesión deberá enviarse a todos los miembros del Consejo, con un plazo máximo de cinco días  desde su aprobación por el consejo, además de ser publicada en la página web del Centro de Alumnos bajo responsabilidad del Secretario General.
			\end{art}

	\section{Funcionamiento y apartados}

		\begin{art}\label{definicionApartados}
			Los Estatutos del Centro de Alumnos de Ingeniería pueden ser complementados con Reglamentos y Protocolos para regular formalmente aspectos particulares del CAi y de la vida en la Escuela. Estos Apartados Oficiales constituirán junto a los Estatutos el denominado Cuerpo de Estatutos, el cual funcionará de la siguiente forma:
			\begin{enumerate}
				\item Todo Apartado deberá definirse en uno o más Artículos que se encontrarán bajo un titular propio en la sección Funcionamiento y Apartados de los Estatutos principales. Respectivamente, cada uno de ellos hará referencia a un documento anexo con carácter de Apartado Oficial.
				\item Los Estatutos deberán ser siempre publicados junto a sus Apartados.
				\item Existen dos tipos de Apartados: Los Reglamentos son una extensión de este Estatuto principal, por lo que deberá velarse por su cumplimiento de igual forma que el mismo. Los Protocolos definen directrices que guían el comportamiento de los Organismos del CAi. Su cumplimiento es fiscalizable para contribuir a una mejor gestión, pero por sí solo no es vinculante a un motivo de cesación del cargo de acuerdo al artículo \aref{ceseCargo}.
				\item Los Apartados deberán ser creados y modificados según la forma estipulada para este caso en el \aref{normaReforma}.
				\item Los Apartados están bajo jurisdicción de los Estatutos, en cuanto deben respetar y cumplir los principios y disposiciones que aquí se enuncian. No pueden interferir en materias básicas para las cuales este documento se pronuncie, en cuyo caso primará siempre lo señalado en los Estatutos principales.
			\end{enumerate}
		\end{art}

		\subsection*{Sobre el Tribunal Calificador de Elecciones}

			\begin{art}\label{definicionTRICEL}
				El Tribunal Calificador de Elecciones, en adelante TRICEL, será el organismo encargado de implementar y supervisar todo proceso eleccionario o de sufragio universal a nivel de CAi. Se regirá por un Apartado propio denominado "Reglamento TRICEL CAi", en el cual se especifican sus funciones y atribuciones durante los sufragios. El TRICEL tendrá carácter temporal, cumpliendo sus funciones sólo respecto del proceso eleccionario para el cual se constituye. Se conformará en el siguiente momento:
				\begin{enumerate}
					\item En el caso de elecciones de representantes del CAi, el mismo día hábil en que comience el periodo de inscripción de candidatos.
					\item Para cualquier otro tipo de plebiscito, consulta directa o sufragio universal, lo hará en la misma sesión en que éste se convoque.
				\end{enumerate}
			\end{art}

			\begin{art}\label{funcionesTRICEL}
				Las funciones del TRICEL son:
				\begin{enumerate}
					\item Velar por el cumplimiento de las normas estatutarias y reglamentarias en el respectivo proceso eleccionario. Deberá discutir y aplicar su criterio para discutir ante disyuntivas que puedan existir.
					\item Conocer de cualquier asunto relacionado con la elección para la cual se constituye.
					\item Complementar lo referente a la mecánica de votación y elección en forma de reglamento instructivo.
					\item Determinar la lista de personas con derecho a voto de acuerdo a este estatuto.
					\item Distribuir el material necesario para la implementación del acto eleccionario y recoger el mismo, luego de realizado el acto.
					\item Atender e investigar los reclamos y observaciones presentados por cualquier estudiante, que haya ejercido su derecho a voto, con respecto a la elección.
					\item Emitir resoluciones frente a los casos o reclamos presentados para su fallo y conocimiento, previa deliberación de sus miembros. Las resoluciones serán apelables por vía de consideración ante el Tribunal.
					\item Calificar la elección dando su dictamen respecto de la legitimidad o nulidad parcial o total del acto.
				\end{enumerate}
			\end{art}

			\begin{art}\label{composicionTRICEL}
				El TRICEL estará compuesto por los siguientes miembros:
				\begin{enumerate}
					\item Cualquier miembro del Comité Ejecutivo, el cual será electo de la forma en que este lo estime conveniente. Dicho miembro presidirá el Tribunal.
					\item Dos delegados de cada  Consejo escogidos por los delegados del respectivo Consejo en una sola votación. En el caso de presentarse un número par de listas se elegirán tres delegados del Consejo Generacional con el objetivo de lograr un número impar de integrantes del TRICEL.
					\item Los apoderados de las listas postulantes al Comité Ejecutivo si corresponde. Ellos deberán ser alumnos regulares. El cargo de miembro del TRICEL es incompatible con el de candidato a una elección para la cual se constituye el Tribunal.
					\item Para la elección de Comité Ejecutivo y Consejeros, un profesor de la Escuela, o en su defecto, uno de los representantes de postgrado elegido entre el Consejero de Postgrado, el Jefe de Investigación y los delegados de postgrado del Comité Generacional.
				\end{enumerate}
			\end{art}

		\subsection*{Sobre el Presupuesto Participativo}

			\begin{art}\label{definicionPParticipativo1}
				El CAi debe considerar en su gestión la aplicación de la metodología de Presupuesto Participativo, la cual consiste en asignar recursos mediante votación directa del alumnado a iniciativas estudiantiles, como las mencionadas en el artículo 2º B y C. El Presupuesto Participativo CAi se define de la siguiente manera:
				\begin{enumerate}
					\item Consiste en un monto igual o mayor al 10\% de los ingresos fijos que reciba el CAi por parte del Decanato. El Comité Ejecutivo será quien proponga el monto exacto, de acuerdo a lo anterior, para ser ratificado por mayoría absoluta del Consejo Generacional, al menos cinco días hábiles antes de iniciarse el proceso de inscripción.
					\item Participarán como votantes sólo los estudiantes de la Escuela de Ingeniería de la Pontificia Universidad Católica de Chile, según se definen en el artículo \ref{porcentajeMinimo}. En cuanto a los proyectos postulantes, la mayoría de su equipo deberá cumplir con la condición anterior.
					\item Los proyectos postulantes deberán estar orientados a cubrir las necesidades e inquietudes del estudiantado.
					\item Sin perjuicio de lo anterior, más especificaciones del proceso serán estipulados en el Apartado denominado \textit{"Reglamento del Presupuesto Participativo CAi"}.
				\end{enumerate}
			\end{art}

			\begin{art}\label{definicionPParticipativo2}
				El Presupuesto Participativo CAi se asignará de acuerdo a los resultados de una elección a realizarse simultáneamente con la de Delegados. El proceso será análogo y paralelo al estipulado en el artículo \aref{definicionesDelegados} y el dinero será exigible por los ganadores de acuerdo a las reglas del proceso. Para ello, serán especificados en el Reglamento del Presupuesto Participativo los siguientes puntos:
				\begin{enumerate}
					\item Los requisitos de postulación, tanto formales como de presentación del proyecto.
					\item Las posibles categorías y la consiguiente subdivisión del monto en porcentajes.
					\item El posible número máximo de proyectos postulantes en las posibles categorías, en cuyo caso se deberá especificar también los criterios de la posible pre-selección en forma clara. En cualquier caso, de existir pre-selección, ésta corresponderá al TRICEL.
					\item El número de proyectos ganadores y la forma de repartición de acuerdo al resultado de la elección.
					\item Las fechas, plazos y metodología para el proceso de entrega de informe(s) y de rendición. En cualquier caso, el proceso de rendición debe cerrarse antes de comenzar el mes de diciembre del año respectivo.
				\end{enumerate}
			\end{art}

			\begin{art}\label{sancionesPParticipativo}
				Cualquier proyecto ganador que no cumpliese con lo estipulado en el artículo \aref{definicionPParticipativo1}, será vetado por tres años, previa aprobación por mayoría absoluta del Consejo Generacional.
				\\
				El veto consiste en que tanto el proyecto como las personas registradas en su equipo no podrán volver a postular un proyecto en ninguno de los tres procesos siguientes.
				\\
				El Reglamento del Presupuesto Participativo contará con una sección especial con las posibles listas de proyectos y personas vetadas, ordenadas por año. Al cierre de cada proceso, el Comité Ejecutivo actualizará los nuevos listados según los ingresos y egresos que corresponda de acuerdo a los criterios anteriores.
			\end{art}

		\subsection*{Sobre el Protocolo Ambiental}

			\begin{art}\label{definicionProtocoloAmbiental}
				El Comité Ejecutivo del CAi realiza actividades y gestiones que tienen un impacto ambiental en la comunidad que lo rodea. Por lo mismo, deberá preocuparse de minimizar este impacto en cualquier circunstancia, en la medida de lo posible según sus facultades y recursos. Para guiar esta preocupación, el CAi contará con un Apartado denominado "Protocolo Ambiental", en el cual se establecerán elementos mínimos a considerar y cumplir.
			\end{art}

		\subsection*{Sobre el Protocolo de Uso de Espacios y Equipos}

			\begin{art}\label{definicionProtocoloEspacios}
				El Comité Ejecutivo del CAi es también responsable del buen uso de los espacios de la comunidad de Ingeniería, lo cual ha sido validado por la Escuela de Ingeniería. Los espacios son de la comunidad y están disponibles para que sus miembros los utilicen, pero siempre en respeto de los demás participantes de la misma. Para guiar el buen uso de los espacios y la responsabilidad de asignación, el CAi contará con un Apartado denominado "Protocolo de Uso de Espacios", en el cual se especificará y comunicará los procesos, requisitos y plazos para la solicitud y gestión de espacios.
			\end{art}

		\subsection*{Sobre el Protocolo de Transparencia}

			\begin{art}\label{definicionProtocoloTransparencia}
				El CAi es una institución que busca realizar en sus diferentes organismos una gestión responsable y transparente con sus representados y la comunidad en general. No se entenderá transparencia como la publicación de información privada, sino como la preocupación por disponer abiertamente la información de uso y beneficio público a largo plazo.
				\begin{enumerate}
					\item Los Delegados, Consejeros Académico y de Postgrado y el Comité Ejecutivo deberán publicar Cuentas Públicas dos veces en su periodo por medios tanto físicos como virtuales, ambos accesibles a sus representados. Cada una de ellas debe sintetizar respectivamente toda su gestión hasta el momento.
					\item En el caso del Comité Ejecutivo, deberá llevar un registro de las gestiones y actividades que desee llevar a cabo. En este registro deberá existir una preocupación por conservar la información relevante, tanto si es específica para sus sucesores como general para la comunidad. Además, deberá agregar al registro mediciones respecto a los resultados de estas gestiones.
				\end{enumerate}
				Sin perjuicio de los puntos anteriores, el CAi contará con un Apartado denominado "Protocolo de Transparencia", en el cual se especificará los elementos, formas y criterios que guían una gestión transparente.
			\end{art}

	\section{Reformas}\label{reformas}

		\begin{art}
			Estos estatutos son de carácter rígido e inapelable, es decir, no se aceptan excepciones, solo reformas.
		\end{art}

		\begin{art}\label{normaReforma}
			Cualquier posible reforma al presente Cuerpo de Estatutos deberá seguir las siguientes normas:
			\begin{enumerate}
				\item Deberá ser tratada, discutida y votada ampliamente en el Consejo Académico y el Consejo Generacional.
				\item  Podrá ser presentada al Consejo Académico y al Consejo Generacional solo a través de miembros de él.

				\item  En caso de ser los Estatutos principales, requerirá acuerdo de a lo menos dos tercios del total de los votos posibles del Consejo Académico y también del Consejo Generacional, independiente de ausencias o presencias. Estas reformas entrarán en vigencia a partir del primer día hábil del semestre académico siguiente a la publicación de ésta.

				\item  En caso de ser un nuevo Apartado o una reforma a uno ya existente, requerirá acuerdo de la mayoría absoluta del total de los votos posibles del Consejo Académico y del Consejo Generacional. Estas reformas entrarán en vigencia a partir de un mes después de la publicación de ésta.
				\item  El Secretario General tendrá la responsabilidad de publicar los cambios respectivos y de actualizar el presente Cuerpo de Estatutos a más tardar dos días hábiles una vez aprobado.
			\end{enumerate}
		\end{art}

	\vfill
	\textsc{Consejo Académico y Consejo Generacional}\\
	Escuela de Ingeniería, Pontificia Universidad Católica de Chile\\
	Santiago, 18 de noviembre de 2015

	\newpage

	\begin{sloppypar}
		\textsc{Elaborado por el Consejo de Escuela 2006}\\
		Fernando Zavala Guzmán, Presidente CAi UC 2006\\
		Felipe Herrera Barros, Consejero Académico CAi UC 2006.

		\textsc{Estatutos aprobados por el Consejo de Escuela 2009}\\
		Pablo Varas Valenzuela, Presidente CAi UC 2009\\
		Sofía Undurraga Pellegrini, Consejera Académica CAi UC 2009\\

		\textsc{Estatutos modificados por el Consejo de Escuela 2010}\\
		Matías Navarro Sudy, Presidente CAi UC 2010\\
		Federico Rodríguez de Castro, Consejero Académico CAi UC 2010

		\textsc{Estatutos modificados por el Consejo de Escuela 2011}\\
		Pablo Vial Birrell, Presidente CAi UC 2011\\
		Juan Pablo Vigneaux Ariztía, Consejero Académico CAi UC 2011\\

		\textsc{Estatutos modificados por el Consejo de Escuela 2012}\\
		Eduardo Toro Nahmías, Presidente CAi UC 2012\\
		María Alejandra Cuevas de la Fuente, Consejera Académica CAi UC 2012\\

		\textsc{Estatutos modificados por el Consejo de Escuela 2012}\\
		Roberto Flores Flores, Presidente CAi UC 2013\\
		Mariana Valle Eguren, Consejera Académica CAi UC 2013\\

		\textsc{Estatutos modificados por el Consejo de Escuela 2014}\\
		Daniel Gajardo Orellana, Presidente CAi UC 2014\\
		Felipe Huerta Pérez, Consejero Académico CAi UC 2014\\

		\textsc{Estatutos modificados por el Consejo de Escuela 2014}\\
		Gonzalo Jara Saba, Presidente CAi UC 2015\\
		Rosario Contesse Blanc, Consejera Académica CAi UC 2015\\

		\textsc{Estatutos modificados por el Consejo Generacional y el Consejo Académico 2015}\\
		Gonzalo Jara Saba, Presidente CAi UC 2015\\
		Rosario Contesse Blanc, Consejera Académica CAi UC 2015\\

		\textsc{Estatutos modificados por el Consejo Generacional y el Consejo Académico 2015}\\
		Tomás Ramírez Sarmiento, Presidente CAi UC 2016\\
		Joaquín Rodríguez Landaeta, Consejero Académico CAi UC 2016\\

		\textsc{Estatutos modificados por el Consejo Generacional y el Consejo Académico 2016}\\
		Tomás Ramírez Sarmiento, Presidente CAi UC 2016\\
		Joaquín Rodríguez Landaeta, Consejero Académico CAi UC 2016\\

		\textsc{Estatutos modificados por el Consejo Generacional y el Consejo Académico 2017}\\
		Josefina Calonge Gilardoni, Presidente CAi UC 2017\\
		Josefina Salas Kantor, Consejera Académica CAi UC 2017\\

		\textsc{Estatutos modificados por el Consejo Generacional y el Consejo Académico 2017}\\
		Javiera Rivera Olivares, Presidente CAi UC 2018\\
		Isidora Vizcaya Soto, Consejera Académica CAi UC 2018\\

		\textsc{Estatutos modificados por el Consejo Generacional y el Consejo Académico 2018}\\
		Javiera Rivera Olivares, Presidente CAi UC 2018\\
		Isidora Vizcaya Soto, Consejera Académica CAi UC 2018\\

		\textsc{Estatutos modificados por el Consejo Generacional y el Consejo Académico 2018}\\
		Javiera Rivera Olivares, Presidente CAi UC 2018\\
		Isidora Vizcaya Soto, Consejera Académica CAi UC 2018\\

		%\textsc{Estatutos modificados por el Consejo Generacional y el Consejo Académico 2018}\\
		%Paolo Fabia Vandatta, Presidente CAi UC 2019\\
		%Ángela Parra Martinez, Consejera Académica CAi UC 2019\\

	\end{sloppypar}
\end{document}
